\newif\ifshowsolutions
\showsolutionstrue
\input{../preamble}



%%%%%%%%%%%%%%%%%%%%%%%%%%%%%%
% HEADER
%%%%%%%%%%%%%%%%%%%%%%%%%%%%%%

\chead{
  {\vbox{
      \vspace{2mm}
      \large
      Machine Learning \& Data Mining \hfill
      Caltech CS/CNS/EE 155 \hfill \\[1pt]
      Set 4\hfill
      January 2019. \\
    }
  }
}

\begin{document}
\pagestyle{fancy}



%%%%%%%%%%%%%%%%%%%%%%%%%%%%%%
% PROBLEM 1
%%%%%%%%%%%%%%%%%%%%%%%%%%%%%%

\newpage
\section{Deep Learning Principles [35 Points]}
\materials{lectures on deep learning}
 
\begin{problem}[10]
  Backpropagation and Weight Initialization
\end{problem}

\subproblem[5]

\begin{subsolution}

\end{subsolution}

\subproblem[5]

\begin{subsolution}

\end{subsolution}



\problem \textbf{[10 Points]}


\begin{solution}

\end{solution}



\problem Approximating Functions \textbf{[15 Points]}

\subproblem[7]

\begin{subsolution}

\end{subsolution}

\subproblem[8]

\begin{subsolution}

\end{subsolution}
 

% problem 2
\newpage
\section{Depth vs Width on the MNIST Dataset  [25 Points]}

\problem \textbf{Installation} \textbf{[2 Points]}

\begin{solution}

Keras:

Tensorflow:

\end{solution}


\problem \textbf{The Data} \textbf{[3 Points]}

\subproblem[1]

\begin{subsolution}

\end{subsolution}

\subproblem[2]
 
 \begin{subsolution}


\end{subsolution}
 
 \problem \textbf{Modeling Part 1} \textbf{[8 Points]}

\begin{solution}

\end{solution}
 
 \problem \textbf{Modeling Part 2} \textbf{[6 Points]}
 
 \begin{solution}

\end{solution}
 
  \problem \textbf{Modeling Part 3} \textbf{[6 Points]}
 
  \begin{solution}

\end{solution}
 
 \newpage
 % problem 3
 \section{Convolutional Neural Networks  [40 Points]} 
 \problem Zero Padding \textbf{[5 Points]}

\begin{solution}

\end{solution}

\problem[5] 5 x 5 Convolutions


\subproblem[2]

\begin{subsolution}

\end{subsolution}

\subproblem[3] 

\begin{subsolution}

\end{subsolution}
 
\problem[10]

\subproblem[3]

\begin{subsolution}

\end{subsolution}

\subproblem[3]

\begin{subsolution}

\end{subsolution}

\subproblem[4]

\begin{subsolution}

\end{subsolution}

\problem[20] 

\subproblem[6, each 2]

\begin{subsolution}

\end{subsolution}

\end{document}
